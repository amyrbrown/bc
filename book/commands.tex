% define page size and margins etc
\geometry{
  paperwidth=18.91cm,paperheight=24.589cm,
  vmargin=1.9cm, % top and bottom margins
  inner=1.9cm, % inside margin
  outer=2.29cm, % outside margin
  bindingoffset=0.89cm % gutter
} 

% define headers (empty) and footers
% fiddle with chaptermark so we can make it not all caps
\renewcommand{\chaptermark}[1]{\markboth{#1}{}}
\renewcommand{\sectionmark}[1]{\markright{#1}{}}

% next get rid of existing header and footer and header rule
\fancyhead{}
\fancyfoot{}
\renewcommand{\headrulewidth}{0pt}

% now make the footer the way we want it:
% page number on right side of footer for odd pages 
\rfoot[]{\small{\textsf{\thepage}}}
\fancyhfoffset[EL]{0cm} % this looks like it doesn't do anything, but
                        % it seems to remind it to line up headers with the
                        % rest of the text

% page number and chapter name on left side of footer for even pages 
\lfoot[\small{\textsf{\thepage \hspace{0.25cm} \leftmark}}]{}

% make plain pages have no headers or footers
\fancypagestyle{plain}{\fancyhf{}}

% This makes sure there are no headers/footers on empty pages before 
% a new chapter, if the prev chapter ends on right side
\makeatletter
\def\cleardoublepage{\clearpage\if@twoside \ifodd\c@page\else
    \hbox{}
    \thispagestyle{plain}
    \newpage
    \if@twocolumn\hbox{}\newpage\fi\fi\fi}
\makeatother \clearpage{\pagestyle{plain}\cleardoublepage}


% set all section headers to be sans serif
\allsectionsfont{\normalfont\sffamily}

% format the table of contents
\renewcommand{\cftchapfont}{\sffamily}     % set TOC entries to sserif
\renewcommand{\cftchappagefont}{\sffamily} % set TOC page numbers to sserif

% make all verbatim (code blocks) text smaller, just because it was bugging me
\let\oldverbatim\verbatim
\renewcommand\verbatim{\small\oldverbatim}


% format title of TOC: make sure this matches chapter head format as set below
\renewcommand{\cfttoctitlefont}{\hfill\Huge\sffamily} 

\setcounter{tocdepth}{0} % sets what level of header is shown in the TOC
\setcounter{secnumdepth}{1} % sets what level of subsect. are numbered

% introduce penalty for widows and orphans (can increase to 10 000, although
% not recommended)
% upped the widow/orphan penalty to 600 from 300; seems to have removed
% all widows and orphans -- ARB Mar 30 2012
\widowpenalty=600
\clubpenalty=600

\newenvironment{swcdescription}
{\begin{description}[itemsep=-0.1ex,parsep=0.3ex,topsep=0.2ex,leftmargin=10mm]}
{\end{description}}

\newenvironment{swcenumerate}
{\begin{enumerate}[itemsep=-0.8ex,topsep=0.2ex,leftmargin=10mm]}
{\end{enumerate}}

% new environment for second-level nested enumerated lists
\newenvironment{swcenumerate2}
{\begin{enumerate}[itemsep=-0.7ex,topsep=-1ex,leftmargin=10mm]}
{\end{enumerate}}

\newenvironment{swcitemize}
{\begin{itemize}[itemsep=-0.8ex,topsep=0.2ex,leftmargin=8.5mm]}
{\end{itemize}}

% new environment for second-level nested itemized lists
\newenvironment{swcitemize2}
{\begin{itemize}[itemsep=-0.5ex,topsep=-1ex,leftmargin=9mm]}
{\end{itemize}}

% new environments for objectives and keypoints
\newenvironment{objectives}{\subsubsection*{Objectives}}{}
\newenvironment{keypoints}{\subsubsection*{Key Points}}{}
\newenvironment{challenge}{\subsubsection*{Challenge}}{}

% new environment for callout boxes
\newenvironment{swcbox}[1]{\begin{quote}\noindent\textbf{#1}}{\end{quote}}

%% Make the space above and below captions smaller
\setlength{\abovecaptionskip}{1.2ex}
\setlength{\belowcaptionskip}{-1.5ex}

% URLs go in footnotes.
\newcommand{\urlfoot}[2]{{#2}\footnote{#1}}

% Glossary reference.
\newcommand{\gl}[2]{\emph{#1}}

% figures and references
\newcommand{\figref}[1]{Figure~\ref{#1}}
\newcommand{\swcgraphics}[4]{\begin{figure}\begin{center}\includegraphics[scale=#4]{#3}\caption{#2}\label{#1}\end{center}\end{figure}}

% SQL output tables
\newenvironment{sqltable}[1]{\begin{tabular}{#1}}{\end{tabular}}

% glossary terms
\newcommand{\gterm}[1]{\noindent\textbf{#1}}

% inline code (shrink because of our choice of fonts)
\newcommand{\code}[1]{{\footnotesize\texttt{#1}\normalsize}}

% Environments for code listings
% The forms ending in 'X' set the style, but should not be used directly.
% The forms that don't end in 'X' append a micro-vertical space to the end,
% then check if the previous construct had that space: if so, they suppress
% the inter-block spacing so that consecutive blocks aren't separated by
% inter-paragraph-sized blanks.
\DefineVerbatimEnvironment{VerbInX}{Verbatim}{fontsize=\relsize{-1}}
\DefineVerbatimEnvironment{VerbOutX}{Verbatim}{fontshape=it,fontsize=\relsize{-1}}
\DefineVerbatimEnvironment{VerbErr}{Verbatim}{fontshape=it,fontseries=b,fontsize=\relsize{-1}}
\DefineVerbatimEnvironment{VerbFile}{Verbatim}{fontsize=\relsize{-1}}

\newenvironment{VerbIn}
  {\VerbatimEnvironment
   \par\VerbInX}
  {\endVerbInX\vskip1sp}

\newenvironment{VerbOut}
  {\VerbatimEnvironment
   \par\ifdim\lastskip=1sp\vspace{\dimexpr-2\topsep-2\partopsep\relax}\fi
   \VerbOutX}
  {\endVerbOutX}

\raggedbottom
